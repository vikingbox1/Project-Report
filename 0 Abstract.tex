Quantum computing promises revolutionary advances across various fields, including cryptography, complex simulations, and optimisation problems. However, practical implementation faces substantial hurdles, many of which arise from challenges in quantum error correction (QEC), particularly the high resource overhead and gate complexity in fault-tolerant computations. Existing methods for optimising logical Clifford operations in stabiliser codes often suffer from computational inefficiency or limited scalability. To address this, this thesis introduces an algorithmic approach based on a theory by Dr. Mark Webster. Our goal is to reduce the complexity of logical Clifford operations by finding implementations that can be decomposed into SWAPs, single-qubit Cliffords, and as few two-qubit transvections as possible. The solution employs an evolutionary search framework, combining elitism, tournament selection, adaptive mutation, and parameter tuning to efficiently navigate the solution space. The methodology was rigorously tested on benchmark stabiliser codes such as [[5,1,3]] and [[7,1,3]], and applied to the topologically significant [[8,2,2]] Toric code to showcase its ability to discover previously undocumented logical Clifford gate implementations. Key achievements include an efficient constant-depth implementation of the logical S gate and relatively low-depth versions of H, CZ, and CNOT gates. Quantitative results show a high success rate of up to \(86\%\) for smaller codes \(n<7\). To the best of our knowledge, this is the first reported approach to identify logical Clifford operators for general [[n,k,d]] stabiliser codes, underscoring its potential for improving implementations in both familiar and unexplored codes. Future work could integrate machine learning to further improve adaptability and performance on larger, more complex quantum codes.
