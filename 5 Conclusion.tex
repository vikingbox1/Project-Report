\chapter{Conclusion} \label{Chap6}
This thesis has presented an effective and scalable method for optimising logical Clifford operations in stabiliser quantum error-correcting codes through an evolutionary search algorithm. By minimising two-qubit transvections, the algorithm addresses one of the key practical limitations in fault-tolerant quantum computing, circuit depth and gate-induced error rates. Experimental validation across various stabiliser codes, including the Toric code, demonstrates the algorithm's ability to discover, low-depth implementations of Clifford gates such as \(S\), \(H\), \(CZ\), and \(CNOT\).

The findings show promising results, with success rates up to \(86\%\) on benchmark codes and significant simplifications in circuit design, especially for the logical S gate. Compared to existing techniques, the proposed method provides a more generalisable and efficient route to circuit optimisation, particularly for codes not previously studied in depth.

While the approach has shown strong performance on smaller codes, its full potential will be realised with greater computational resources and enhanced algorithmic strategies, such as machine learning-guided mutation. These advancements will not only enable application to larger codes but also pave the way for practical deployment in real-world quantum hardware. Ultimately, this research contributes a flexible and robust framework to support the ongoing pursuit of fault-tolerant, scalable quantum computation.
